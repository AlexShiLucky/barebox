\section{MDNS client}

% Short description/overview of module functions
This module can take an mDNS hostname and announce it to the network. It will defend the name when
a peer tries to claim it. It can get the IP address of other peers using their hostname, as well as their hostname from the IP address.
mDNS answers are cached and called upon when a request is matched until the TTL has expired.

\subsection{pico$\_$mdns$\_$init}

\subsubsection*{Description}
Function to initialise the mDNS client with a given hostname.
\subsubsection*{Function prototype}
\begin{verbatim}
int pico_mdns_init(char *hostname, void (*cb_initialised)(char *str, void *arg), void *arg);
\end{verbatim}

\subsubsection*{Parameters}
\begin{itemize}[noitemsep]
\item \texttt{hostname} - Hostname to be used with MDNS. Should end with '.local'.
\item \texttt{cb$\_$initialised} - Callback function that is called when the initialisation process is done. The status variable indicates wheter the synchronisation was succesfull or not.
\item \texttt{arg} - Argument for callback supplied by user. This can be used if you want to pass some variable into your callback function.
\end{itemize}

\subsubsection*{Return value}
On success, this call returns 0 if the synchronisation operation has succeeded.
On error, -1 is returned and \texttt{pico$\_$err} is set appropriately.

\subsubsection*{Errors}
\begin{itemize}[noitemsep]
\item \texttt{PICO$\_$ERR$\_$EINVAL} - invalid argument
\item \texttt{PICO$\_$ERR$\_$ENOMEM} - not enough space
\end{itemize}

\subsubsection*{Example}
\begin{verbatim}
pico_mdns_init("my_name.local", &callback, NULL);
\end{verbatim}


\subsection{pico$\_$mdns$\_$getaddr}

\subsubsection*{Description}
Function to get the address of a certain hostname or url.

\subsubsection*{Function prototype}
\begin{verbatim}
int pico_mdns_getaddr(const char *url, void (*callback)(char *ip, void *arg), void *arg);
\end{verbatim}

\subsubsection*{Parameters}
\begin{itemize}[noitemsep]
\item \texttt{url} - Hostname of which the address will be retreived.
\item \texttt{callback} - Callback function that is called when the address is retreived.
\item \texttt{arg} - Argument for callback supplied by user. This can be used if you want to pass some variable into your callback function.
\end{itemize}

\subsubsection*{Return value}
On success, this call returns 0 if the time is set.
On error, -1 is returned and \texttt{pico$\_$err} is set appropriately.

\subsubsection*{Example}
\begin{verbatim}
int ret = pico_mdns_getaddr("other_host.local", callback, NULL);
\end{verbatim}



\subsection{pico$\_$mdns$\_$getname}

\subsubsection*{Description}
Function to get the name of a host using the IP address.

\subsubsection*{Function prototype}
\begin{verbatim}
int pico_mdns_getname(const char *ip, void (*callback)(char *url, void *arg), void *arg);
\end{verbatim}

\subsubsection*{Parameters}
\begin{itemize}[noitemsep]
\item \texttt{ip} - IP address of the host you want the name of.
\item \texttt{callback} - Callback function that is called when the hostname is retreived.
\item \texttt{arg} - Argument for callback supplied by user. This can be used if you want to pass some variable into your callback function.
\end{itemize}

\subsubsection*{Return value}
On success, this call returns 0 if the time is set.
On error, -1 is returned and \texttt{pico$\_$err} is set appropriately.

\subsubsection*{Example}
\begin{verbatim}
pico_mdns_getname("192.168.1.7", &callback, NULL);
\end{verbatim}


\subsection{pico$\_$mdns$\_$getaddr6}

\subsubsection*{Description}
Function to get the IPv6 address of a certain hostname or url.

\subsubsection*{Function prototype}
\begin{verbatim}
int pico_mdns_getaddr6(const char *url, void (*callback)(char *ip, void *arg), void *arg);
\end{verbatim}

\subsubsection*{Parameters}
\begin{itemize}[noitemsep]
\item \texttt{url} - Hostname of which the IPv6 address will be retrieved.
\item \texttt{callback} - Callback function that is called when the address is retrieved.
\item \texttt{arg} - Argument for callback supplied by user. This can be used if you want to pass some variable into your callback function.
\end{itemize}

\subsubsection*{Return value}
On success, this call returns 0 if the time is set.
On error, -1 is returned and \texttt{pico$\_$err} is set appropriately.

\subsubsection*{Example}
\begin{verbatim}
int ret = pico_mdns_getaddr6("other_host.local", callback, NULL);
\end{verbatim}



\subsection{pico$\_$mdns$\_$getname6}

\subsubsection*{Description}
Function to get the name of a host using its IPv6 address.

\subsubsection*{Function prototype}
\begin{verbatim}
int pico_mdns_getname6(const char *ip, void (*callback)(char *url, void *arg), void *arg);
\end{verbatim}

\subsubsection*{Parameters}
\begin{itemize}[noitemsep]
\item \texttt{ip} - IPv6 address of the host you want the name of.
\item \texttt{callback} - Callback function that is called when the hostname is retreived.
\item \texttt{arg} - Argument for callback supplied by user. This can be used if you want to pass some variable into your callback function.
\end{itemize}

\subsubsection*{Return value}
On success, this call returns 0 if the time is set.
On error, -1 is returned and \texttt{pico$\_$err} is set appropriately.

\subsubsection*{Example}
\begin{verbatim}
pico_mdns_getname6("fc00:aaaa:bbbb:cccc::", &callback, NULL);
\end{verbatim}

\subsection{pico$\_$mdns$\_$flush$\_$cache}

\subsubsection*{Description}
Function to flush the mDNS resource record cache.

\subsubsection*{Function prototype}
\begin{verbatim}
int pico_mdns_flush_cache(void);
\end{verbatim}

\subsubsection*{Return value}
This call returns 0.
